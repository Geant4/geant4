\documentclass{hep99}
%\usepackage{epsfig}
\usepackage[dvips]{graphicx}
\begin{document}
\title{Object-Oriented Reconstruction and Analysis for
Helsinki Silicon Beam Telescope}
\author{M.~Haapakorpi, A.~Heikkinen, J.~Heinonen, V.~Karim\"aki,
J.~Klem and T.~Lamp\'en}
\address{Helsinki Institute of Physics, University of Helsinki,
Finland\\[3pt]
E-mail: {\tt Aatos.Heikkinen@cern.ch}}

\abstract{Object-oriented programming and related new software
technologies have been applied to the reconstruction and analysis of
events recorded by Helsinki Silicon Beam Telescope (SiBT) in the
CERN H2 testbeam.
Experience obtained on these new tools and methods is reported.}
\maketitle
\section{Introduction}
As the computer industry has moved
to the object-oriented (OO) technology era, the high energy physics
community is also following the trend.
After years of R\&D and feasibility studies the object technology is now
accepted by the whole HEP community.

We present here how object-based software technologies can be applied
to the reconstruction and analysis of events recorded by SiBT 
detector\cite{sibt}
in the CERN H2 beam. 
This beam is used by Compact Muon Solenoid (CMS)
\cite{cms} collaboration to test detector prototypes. New software is
implemented as part of ORCA (Object-oriented Reconstruction for Cms
Analysis) testbeam software.

The structure of this paper is as follows:
section~\ref{sec:tools} lists the tools used in software development,
section~\ref{sec:design} discusses the software design,
section~\ref{sec:implementation} describes implementation issues and
section~\ref{sec:conclusion} provides critical overview of tools used.

\section{Tools for Object-Oriented Programming\label{sec:tools}}
%\subsection{Starting off: title, author, address and abstract}

The object-oriented programming  paradigm, together with tools from
LHC++ library (Libraries for HEP Computing) \cite{lhc++},
has been used in all aspects of our work. 
The Unified Modelling Language (UML) CASE tool Rational Rose was used
for software design.

We have used Sun WorkShop development environment. WorkShop makes
complex development tasks much easier by providing a tightly
integrated development environment for building, editing, source
browsing, and debugging. 

Histo-Scope package was used for both online and off-line analysis
since it is small enough for online applications.

For the detailed coding of mathematical algorithms we 
used CLHEP (A Class Library for High Energy Physics) \cite{lhc++}. 
We also fully employ the generic programming capability of the Standard
Template Library (STL). ObjectSpace implementation of the STL is used. 

Geant4 \cite{geant4} was used for detector description and event simulation.
An object oriented database management system, Objectivity/DB, was used to
store the beam telescope geometry information and reconstructed
events. Finally, IRIS Explorer has been used in data-analysis. Table 1
%\ref{table:list_of_tools} 
summarises the tools used.

\begin{table}
\label{table:list_of_tools}
\begin{center}
\caption{Tools used in SiBT software development.}
\begin{tabular}{ll}
\br
Tool &Use  \\
\mr
LHC++ & General framework \\
ORCA& CMS framework \\
Objectivity/DB& Percistence \\
CVS& Code management \\
UML& Software modelling \\
Rational Rose& Code generation \\
C++& Programming language \\
STL& Generic programming  \\
CLHEP, NAG C& Mathematics \\
WorkShop& Development environment \\
Qt, Histo-Scope& User interface \\
Geant4& Simulation \\
IRIS Explorer& Data analysis \\
\br
\end{tabular}
\end{center}
\end{table}
%------------------------------------------------------------------------
\section{Software Design\label{sec:design}}
The new SiBT reconstruction software was designed
to be highly modular.  Each module has separate compiling, testing and
documentation facilities allowing independent development of different
parts of the project. Here we are following the coding rules adopted
in the CMS community. Figure \ref{fig:uml} represents the general structure
of SiBT analysis software. Configuration
module provides general methods to change parameters during interactive
data analysis.

%-----------------------------------------------------------------------
\section{Implementation\label{sec:implementation}}
The Rational Rose CASE tool was used to generate source code directly
from UML diagrams. All header files are automatically generated. 
As an example Figure \ref{fig:finder} shows part of 
the class definition for trackfinders.

%------------------------------------------------------------------------
\section{Conclusion\label{sec:conclusion}}
We have used UML CASE tool Rational Rose heavily for code
generation. In our experience the implementation architecture and header
files can be conveniently created using this tool. 
We also conclude that reverse engineering in many cases lacks proper
tools and is in some cases not feasible. 

We find Histo-Scope as modern, handy histogramming tool.
We would also like to see
Objectivity/DB to have more mature tools for interactive data
analysis. For interactive data-analysis IRIS Explorer is not
providing the tools we would expect, so as a practical solution
we also used PAW and Mathematica in our analysis.

\begin{thebibliography}{9}
\bibitem{sibt} Eklund C et al. 1999 {\it Nucl. Instrum. Meth.} A {\bf
430} 321
\bibitem{cms} Muller T 1998 {\it Nucl. Instrum. Meth.} A {\bf 408} 119
\bibitem{lhc++} LCB Status Report/LHC++ 1998 {CERN/LHC 98-11}
\bibitem{geant4} GEANT4: LCB Status Report/RD44 1998 {\it CERN/LHCC-98-44} 

\end{thebibliography}

%-----------------------------------------------------------------------

\begin{figure}
\begin{center}
%\includegraphics[width=2.5in]{uml.ps}
\end{center}
\caption{Modular structure of SiBT analysis software.}
\label{fig:uml}
\end{figure}

%If the figure is too large to fit in a column a * is added after
%\verb"figure" to use the \verb"figure*" environment. Figures should be
%centred in the column or page.


\begin{figure}
\begin{center}
%\includegraphics[width=3in]{finder.ps}
\end{center}
\caption{Trackfinder classes.}
\label{fig:finder}
\end{figure}

\end{document}





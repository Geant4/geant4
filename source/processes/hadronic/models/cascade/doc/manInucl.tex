%\documentstyle[epsf,twoside,fleqn,espcrc2]{article}
%\def \AATOS {/afs/cern.ch/user/m/miheikki/public/html/}
\def \AATOS {/home/miheikki/public/html/}
\def \PIC {/afs/cern.ch/user/m/miheikki/public/html/inucl/doc/}
\def \PIC {/home/miheikki/public/html/inucl/doc/}
% put your own definitions here:
%   \newcommand{\cZ}{\cal{Z}}
%   \newtheorem{def}{Definition}[section]
%   ...
%\newcommand{\ttbs}{\char'134}
%\newcommand{\AmS}{{\protect\the\textfont2
%  A\kern-.1667em\lower.5ex\hbox{M}\kern-.125emS}}


% declarations for front matter
%\title{HETC/INUCL++ manual - System for Modelling Hadronic Intranuclear Processes}

%\author{Aatos Heikkinen, Nikita Stepanov 
%\address{ Helsinki Institute of Physics,
%P.O.Box 9,\\
%FIN-00014 University of Helsinki, Finland }
%	}
  
%\begin{document}

%\begin{abstract}
%HETC/INUCL is a simulation package and a model of intermediate energy nuclear
%reactions.
%INUCL consists of cascade, fission models, pre-equilibrium, and evaporation
%This manual describes the models used in INUCL++. 
%This documet works as an manual, status report and working document for INUCL project. In accordance with this, we represent results acieved with INUCL. Comparison with measurement is made. 
%We also descripe staus of Geant4 installation on INUCL software \cite{titarenko99a}.
%\end{abstract}
%\maketitle

\section{VOCABULARY}
\begin{itemize}
\item {\it NR} Nuclear reaction
\item {\it quasi-elastic reaction} NR where nucleon can be knocked out of its nuclear. 
\item {\it T-nucleus} Target nucleus 
\end{itemize}

\section{INTRODUCTION TO INTERMEDIATE-ENERGY NUCLEAR PHYSICS}


Fig :::. Schematic nucleon structure. 
The central nucleon is shown as collection of three constituent quarks interacting via the color force. 
Closer views show a more complex picture.

At higher energies ($~200~MeV$) energy is high enough to momentarily exite a nucleon into the exited dalta-resonant state ($\Delta$-resonance).

\subsection{Classification of Cross-Sections of NRS}

Let us denote an incident particle and a nucleus under investigation by the symbols $a$ and $A$, respectively.
Following final states following from $a + A$ are possible \cite{iljinov94}:

\begin{itemize}
\item $a + A$ :  elastic scattering
\item $a^{*} + A$ :  scattering with an exitation of the $a$ particle
\item $a + A^{*}$ : scattering with an exitation of the nucleus
\item $a^{*} + A^{*}$ : scattering in which both the particle and the nucleus are exited
\item $a^{'} + A^{'}$ : reaction involving the production of a new particle and a new nucleus
\item $a^{'} + a^{''} + A^{'}$ : reaction involving the production of two new particles and a new nucleus
\end{itemize}

\subsection{Quantum Formulation of Collision Problem}

\subsubsection{Reaction Channels}

Consider a collision of two complex particles. ::: \cite{iljinov94}

\subsubsection{Transition Probabilities and S-Matrix}

In accordance with general priciples of quantum theory, the probability 
 ::: \cite{iljinov94}

\subsubsection{S-Matrix and Cross-SectionsScattering Amplitude}

 ::: \cite{iljinov94}


\subsection{Nuclear Structure Phase Diagram}

THe atomic nucleus with its $A$ nucleons is governed by a large number of degrees of freedom.

\subsection{Nuclear density}

Nuclear charge densities are usually well described using present-day effective two-body forces; it is also clear that saturation of the charge dencity indeed occurs. 
The central density barely varies when the nucleon number changes.
\subsection{Nuclear radius}

\subsection{Mass formula}

A parametrization of the nuclear binding energy in the groud state,
was first discussed by Bethe, Bacher and Weizs\"{a}cker.
Parametrization conatains volume, surface, Coulomb and symmetry correction terms (we neglegt here typical shell model correction terms) and reads

$$B(E, A) = a_{\nu} A - a_{s} A^{2/3} - a_{c} Z(Z-1)A^{-1/3} -$$ 
\begin{equation}
  a_{A}\frac{(A-2Z)^2}{A}
\end{equation}

Nuclear charge radius turns out to be a rather well-defined quantity.

\subsection{Symmetry consepts in Nuclear Physics}
The nuclear two-body force obeys quite a large number of basic invariances.
Forexample invariance under interchange of the spatial coordinates, translation invariance, Galilean invariance, space reflection symmetry, time reversal invariance, rotation invariance in coordinate space and rotation invariance in charge space (isospin).

The symmetry consept (1932) of isospin symmetry, 
describing the charge independence of the nuclear forces by means of the isospin consept with the SU(2) group as the underlying mathematical group was suggested by Heisenberg. 
This (simplest of all dynamical symmetries) expresses the invariance of the Hamiltonian under the exchange of all proton and neutron coordinates \cite{heyde98}.

\subsection{Quantum effects}

On the level of the nucleons themselves, invariance of the total nuclear wave functions under the exchange of identical nucleons affects the possible models realized. 
The Pauli principle implies antisymmetry for the total fermionic wave function and this has very definite consequences for the types of collective motion that can be set up inside the nucleus.


\subsection{INC-Model}

Widely used semiclassical miscroscopic description of a collision between a particle and a nucleus, 
was proposed by Serber \cite{serber47} and Goldberger \cite{goldberger48}.

\section{INTRODUCTION TO INUCL}

\subsection{BACKGROUND}
Originally the INUCL code was published in 1983 \cite{stepanov}.
This code, written with Fortran has been used in [:::] \cite{:::}.

Comprison of results between INCUL and LAHET, CEM95, HETC, CASCADE,
YIELDX, and ALICE code are presented in Refs. \cite{titarenko99a}.

Year 2000 a aroject to implement INUCL into Geant4 hadronic physic
module using C++ wal launched. 

New now  works also as standalone c++ software called INUCL++.

\subsection{SUMMARY OF INUCL MODEL FEATURES}
\begin{itemize}
\item Originally INUCL was designed as a particle - nucleus interaction simulation block for the particle - target interaction simulation program PHOENIX. It produces an exclusive approach to simulating events with reasonable performance.
\item INUCL is based on N. Stepanov Ph.D. thesis, ITEP, Moscow, 1990.
Also, contribution Vladimir D. Kazaritsky.
\item Now we have standalone F77 based INUCL code and INUCL++ written in C++.
INUCL++ is now written using Geant4 coding style and integration to hadronic models in in progress. Problem: speed is five times slower.
\end{itemize}

\subsubsection{INUCL models}
\begin{itemize}
\item Intranuclear cascade
\item Precompound decay (exiton master equation) 
\item Evaporation (Weisskopf - Ewing) 
\item Fission (phenomenological model, incorporating some features of the fission statistical model)
\end{itemize}

\subsubsection{Particles treated}
\begin{itemize}
\item Range of targets allowed arbitrary.
\item Range of projectiles allowed p, n, pi and nuclei.
\item From a few Mev to 10 GeV for n, p, pi and up to about 100 MeV / nucleon for nuclei.
\end{itemize}

\subsubsection{Cross sections}
\begin{itemize}
\item Total inelastic cross section has to be taken from outside to normalize all data. 
\item Total reaction cross-sections [mbarn] were calculated by J.R. Letaw's formulae 

%$45 A^0.7 (1+0.016 sin(5.3-2.63 log10(A)))^(1-0.62 exp(-E / 200) sin(10.9 E^(-0.28)))$
 
\item Ref.: S. Pearlstein, The Astrophysical Journal, 346: 1049-1060, 1989 November 15
\item Fermi energy calculated in a local density approximation.
\end{itemize}

Nuclear density distribution are derived from the Re(Vopt( r)) distribution. In cascade part, nucleus is divided into a finite number of zones with constant density.


%nuclear radius parameterization: By the definition R(A) is derived from eq. Den(R(A)) = 0.01*Den max
\subsubsection{Nucleon nucleon cross-sections}
\begin{itemize}
\item Parametrizations based on the experimental data (ED) are used. 
\item They are energy and isospin dependent. 
\item The parameterizations described in ([1] Barashenkov V.S., Toneev V.D. High Energy interactions of particles and nuclei with nuclei. Moscow, 1972 
%(in Russian, but there is an English translation)) are used.

\item Pauli exclusion in the INC: Simulated particle-particle interaction is accepted only for secondary nucleons which have $E_n > E_f$.

\item Nuclear density effects are recalculated after each step

\item Cascade is stopped when all the particles, which can escape the nucleus, do it. Then conformity with the energy - conservation law is checked and the given event is accepted, if $E_{exitation} > E_{cut} \approx $a few $MeV$.

\item For nucleons binding energies are calculated using mass formula. For pions Vopt is taken to be constant (about 7 MeV).
\end{itemize}

%What criteria for p-h excitation? is the next phase precompound or compound'? Only pions. The next phase is precompound. Initial conditions are defined during the cascad phase: p -number of "particles", i.e. nucleons, which can not escape the nucleus and have too small interaction probability; h - number of "holes" = number of nuclear nucleons involved in the cascade; energy - momentum of the exiton system derived from the conservation law.

\subsubsection{Precompound phase}
%, describe the PE model used, parameters, i.e., partial state densities, transition rates? 
\begin{itemize}

\item Main parameters are taken from (Ribansky I. et al, Nucl.Phys.,1973, A205, p.545 (level densities); Kolbach.C., Z.Phys.,1978, A287, p.319 (matrix elements)). 
%(Only N -> N, N -> N + 2, N -> N -2, N -> N - 1 channels are treated.)
\item The angular distribution is isotropic in the frame of rest of the exiton system.
\end{itemize}

%Describe parameters used: level densities, inverse cross-sections or transmission coefficient, choice of optical model parameters if relevant (or reference to source), range of excitations allowed, inclusive or exclusive results? Weisskopf-Ewing evaporation in competition with fission. Emissions of n,p,d,t,He3,He4,gamma is allowed. Level densities derived from exp.data are used. Angular momentum and spin dependence are not included. Other parameters are the same as in ([1], see 5a.) Fermi breakup is allowed onlyin some extreme cases, i.e. for light nuclei and E(exitation) > 3.*Eb. Only the total nuleus decay into neutrons and protons is treated.


\begin{verbatim}
particles p, n, pi, D, T, He3, He4,\gamma
(INUCLN with neutrino 31.3.98)
pion aborption
interaction crosssections
(all data for (N, N) and (pi, N) interactions (dn/dsigma, d3sigma/d3p, 
            partial multiplicity for npi<=5  error 10-20\%)
pre-equilibrium exiton model 
\end{verbatim}

\section{INUCL MANUAL PAGES}

\subsection{Reaction initial state simulation}
\subsubsection{Projectiles and enery range of model}

The GEANT4 INCUL model is cpable to predict final states :::

The allowed allowed bombarding kinetic enery is recommended to be more than 
$20~MeV$ in the laboratory frame. 
The upper limit of initial kinetic energy is approximately $10~GeV$.

\subsubsection{Nucleus initialization}

\begin{itemize}
\item {\bf Nucleon radii}:
\item {\bf The initial momenta of the nucleons}:
\item {\bf Nucleon binding energy}:
\end{itemize}
\subsection{Random choice of the impact parameter}

The impact parameter $b$ is randomly selected according to
\subsection{Hadron propagation}

\subsection{Selection of particle collisions and particle decays}

\subsection{Hadron interation cross sections}

\subsection{Hadron collision simulation}

\subsubsection{Two-body hadron scattering}
\subsubsection{Angular distribution of two-body hadron scatterings}
\subsubsection{$\pi$-absorption simulation}

\subsection{Resonance decay simulation}

\subsection{Phenomenological potentials}
\subsubsection{The nucleon potential}

\subsection{Pauli blocking simulation}

\subsection{Residual nucleus parameters in the hadron-nucleus collision}

\subsection{Calculation of a residual nucleus exitation energy}

\subsection{Staus of this document}
8.6.02 created by A. Heikkinen

\section{INTRANUCLEAR CASCADE MODEL}

\subsection{NUCLEON DENCITY IN THE ATOM}

Halo nucleus such as $^{11}Li$ are not modelled.

\subsection{IMPULSE DISTRIBUTION}

\subsection{DISTRIBUTION OF POTENTIAL ENERGY}

\subsection{QUANTUM EFFECTS}
Pauli exclusion principle

\subsection{DESCRIPTION OF INC}


\section{FISSION}

\section{PRE-EQUILIBRIUM}

\section{EVAPORATION}
Exited nuclei cools further trough the emission of gamma radiation.
If simulation is detailed,
the emitted gamma rays contain information about the cooling route 
and region the nucleus is passing trough.

\section{FROM INUCL TO INUCL++}

\subsection{DESIGN AND ANALYSIS}
We desided to make separate INUCL++ implementations one as an stand
alone with spesific cross-section data and particle definition, and
another iplementation that re-uses Geant4 classes.
So, requirements for the INUCL++ came mainly from Geant4.
Coding style and organization should follow those used in Geant4.


Interace-classes to Geant4 hadronic models define the separation of
cascade, fission, pre-equilibrium, and evaporation. Thus INUCL++
implementation in Geant4 consists of four separate modules.

\subsection{IMPLEMENTATION}

\subsection{TESTING}
Extensive testing :::

Speed comparison :::

\section{INUCL++ AS A STANDALONE PROGRAM}

\section{INUCL++ IN GEANT4}

\section{RESULTS}
We have rewritten INUCL into C++ and tested the performance.
Here we combine INUCL models and HETC INC.


We give here summary of results. Each configuration consists of N=1000 events simulated with same energy etc. parameters. 

\subsection{Particle multiplicities}
Multiplities for collisions between $p$  projectile ( $0.5~GeV< E <5.0~GeV$) and aluminium $Al$, iron $Fe$ and lead $Pb$ tragets (at rest) are listed in Figs. \ref{pAlMultiplicity} - \ref{pPbPionMultiplicity}.



\begin{figure}
  \begin{center}
    \leavevmode
    \mbox{\epsfxsize=8cm \epsffile{\PIC pAlMultiplicity.eps} }
       \caption{Particle multiplicities for $(p, ^{27}Al$).}
  \label{pAlMultiplicity}
  \end{center}
\end{figure}



\begin{figure}
  \begin{center}
    \leavevmode
    \mbox{\epsfxsize=8cm \epsffile{\PIC pFeMultiplicity.eps} }
       \caption{Particle multiplicities for $(p, ^{56}Fe$).}
  \label{pFeMultiplicity}
  \end{center}
\end{figure}


\begin{figure}
  \begin{center}
    \leavevmode
    \mbox{\epsfxsize=8cm \epsffile{\PIC pPbMultiplicity.eps} }
       \caption{Particle multiplicities for $(p, ^{82}Pb$).}
  \label{pPbMultiplicity}
  \end{center}
\end{figure}


\begin{figure}
  \begin{center}
    \leavevmode
    \mbox{\epsfxsize=8cm \epsffile{\PIC pPbPionMultiplicity.eps} }
       \caption{Pion multiplicities.}
  \label{pPbPionMultiplicity}
  \end{center}
\end{figure}



\subsection{Particle energies}

Figs. \ref{pPbProtonEnergy} - \ref{p500MeVPbProtonEnergy} give an summary of cascade products energy-levels.

\begin{figure}
  \begin{center}
    \leavevmode
    \mbox{\epsfxsize=8cm \epsffile{\PIC pPbProtonEnergy.eps} }
       \caption{Average proton energies.}
  \label{pPbProtonEnergy}
  \end{center}
\end{figure}


\begin{figure}
  \begin{center}
    \leavevmode
    \mbox{\epsfxsize=8cm \epsffile{\PIC pPbNeutronEnergy.eps} }
       \caption{Average neutron energies.}
  \label{pPbNeutronEnergy}
  \end{center}
\end{figure}


\begin{figure}
  \begin{center}
    \leavevmode
    \mbox{\epsfxsize=8cm \epsffile{\PIC pPbNucleonExitation.eps} }
       \caption{Average nucleon exitation energy.}
  \label{pPbNucleonExitation}
  \end{center}
\end{figure}


\begin{figure}
  \begin{center}
    \leavevmode
    \mbox{\epsfxsize=8cm \epsffile{\PIC p50MeVPbProtonEnergy.eps} }
       \caption{Proton energy spectrum, when projectile particle (p) had an energy of $0.05~GeV$.}
  \label{p50MeVPbProtonEnergy}
  \end{center}
\end{figure}


\begin{figure}
  \begin{center}
    \leavevmode
    \mbox{\epsfxsize=8cm \epsffile{\PIC p500MeVPbProtonEnergy.eps} }
       \caption{Proton energy spectrum, when projectile particle (p) had an energy of $0.5~GeV$.}
  \label{p500MeVPbProtonEnergy}
  \end{center}
\end{figure}


\bibliographystyle{unsrt}  % Options plain, unsrt, alpha, abbrv

%\bibstyle{plain}

%\begin{thebibliography}{99}
\bibliography{\AATOS texts/references/references.bib} 
%\end{thebibliography}

\begin{appendix}

\section{PROJECT plan}
\begin{itemize}
\item we will continue testing INUCL++ against previous verison.
\item we reorganize the code is sutch mannet that integratin to Geant4
  is smooth
\item before year 2002 will provide INUCL++ as a stanalone program, 
or as an hadronic cascaded modules 
(INC, pre-equilibrium, evaporation, fission) in Geant4. 
\end{itemize}

\section{UML DIAGRAMS}
Preliminary suggestion for Inucl class interface to Geant4 cascade, pre-equilibrium
and evaporation modules.

\begin{figure}
  \begin{center}
    \leavevmode
    \mbox{\epsfxsize=8cm \epsffile{\PIC cascadeUml.eps} }
       \caption{General class structure for Inucl INC model.}
  \label{massPb}
  \end{center}
\end{figure}

\begin{figure}
  \begin{center}
    \leavevmode
    \mbox{\epsfxsize=8cm \epsffile{preEquilibriumUml.eps} }
       \caption{General class structure for Inucl pre-equilibrium model.}
  \label{massPb}
  \end{center}
\end{figure}

\begin{figure}
  \begin{center}
    \leavevmode
    \mbox{\epsfxsize=8cm  \epsffile{evaporationUml.eps} }
       \caption{General class structure for Inucl evaporation/fission model.}
  \label{massPb}
  \end{center}
\end{figure}

\section{DOWNLOADING INUCL++}
INUCL++ is downloadable from INUCL homepage at
http://cern.ch/aatos.heikkinen/inucl

\scriptsize
\begin{verbatim}

\end{verbatim}
\normalsize

\section{OVERVIEW OF GEANT4 INUCL ACTIVITIES 2001-2002 }

\subsection{General}
Inside LHC Programme Software and Physics project Geant4 activities are under catecory {\it Simulation and event recosntruction}.
In 2001 HIP joined as participating institute the worl wide Geant4 collaboration with the major responsibility for development and maintenance of the nuclear evaporation and intra-nuclear cascade processes. 
Two major codes HETC and INUCL are to be implemented and developed for Geant4 hadronic processes.

The conversion of the hadronic evaporation processes of HETC code to Geant4 was completed in 2001 and good progress was made in the Object-Oriented implementation of the intra-nuclear cascade processes.

For another important nuclear Maonte Carlo code, INUCL, an object oriented model was prepared, containing models for intra-nuclear cascade, pre-equilibrium state, fission and evaporation. 

Year 2002 started with implementation of  INUCL++ as a standalone software.
A preliminary version of architecture using INUCL++ intra-nuclear model. It also integrates HETC cascades and utilites Geant4 hadronic models frameworks.

\subsection{ABOUT GEANT4}
Geant4 represents one of the largest and most ambitious projects of geographically-distributed software development and large-scale object-oriented systems.


\subsection{COLLABORATION TSB AND CB WORK}

In 2001 HIP joined as participating institute the worl wide Geant4 collaboration with the major responsibility for development and maintenance of the nuclear evaporation and intra-nuclear cascade processes. 

Our contribution in TSB and CB will be focused in near future to initiatives and coordination of  hadronic cascade framework inside Geant4 hadronic framework.

\subsection{INCUL}

\subsubsection{INTORODUCTION TO INUCL}
INUCL++ is an object oriented implementation of Monte Carlo transport code INUCL  for computing the properties of high-energy hadronic cascades in matter. 
The functionality of INUCL++ is divided into intranuclear cascade, pre-equilibrium, fission, and evaporation parts. 

The stand-alone version INUCL++ is comparable to codes such as HETC and LAHET \cite{titarenko99}. 

\subsubsection{INUCL++}
We have made independent version  INUCL++ using OO methods.
 
Architecture of INUCL++ intra-nuclear model plays an important role in our Geant4 work, since we use it also for HETC cascades modelling and as an platform to form a general framework for cascading model.

\subsection{PLANS FOR YEAR 2002}

\begin{itemize}
\item INUCL cascade model introduced to Geant4

\item We also aim  to have significant progress on pre-equilibrium, fission models.
\end{itemize}

\begin{figure}
  \begin{center}
    \leavevmode
    \mbox{\epsfxsize=8cm  \epsffile{massPb.eps2} }
       \caption{::: Pb from 1 GeV .}
  \label{massPb}
  \end{center}
\end{figure}

\subsection{HIP Geant4 activities 2001-2002}

General:
\begin{itemize}
\item In 2001 HIP joined as participating institute the world wide Geant4 collaboration
\item Responsibility for development and maintenance of the nuclear evaporation and intra-nuclear cascade processes. 
\item Two major codes HETC and INUCL are to be implemented and developed for Geant4 hadronic processes.
\end{itemize}

Research activities:
\begin{itemize}
\item Participation to Geant4 TSB and CB meetings
\item The conversion of the hadronic evaporation processes of HETC code to Geant4 was completed in 2001
\item Good progress was made in the Object-Oriented implementation of the intra-nuclear cascade processes

\item An object oriented model of INUCL was prepared, containing models for intra-nuclear cascade, pre-equilibrium state, fission and evaporation 
\item Implementation of  INUCL++ as a standalone software was finalized

\item Prototype for cascading framework was prepared using experience gained from INUCL OO design
\end{itemize}

Future plans:
\begin{itemize}
\item INUCL cascade model introduced to Geant4
\item Fully integration of HETC intra nuclear cascade -model and Geant4.
\item Significant progress on HETC and INUCL pre-equilibrium and fission models.
\item Introduction of general cascade framework to Geant4 hadronic models
\end{itemize}

\section{SAMPLE RUN RESULT}
\scriptsize
\begin{verbatim}
Number of events 100000
 average multiplicity 21.5523
 average proton number 3.53569
 average neutron number 14.5048
 average nucleon Ekin 0.0321547
 average proton Ekin 0.0848587
 average neutron Ekin 0.0193075
 average pion number 0.3036
 average pion Ekin 0.132491
 average pi+ 0.1165
 average pi- 0.058
 average pi0 0.1291
 average A 126.891
 average Z 53.1615
 average Exitation Energy 0.0258312
 average num of fragments 1.69988
 fission prob. 0.0117 c.sec 20.592
 =====================================================
 ********** Izotop analysis ******************

 ++++++++++++++++++++++++++++++++++++++++++++++++++++++++
 **** izotop Z **** 0
 A 1 exp.cs 33520 err 12.4
 sim. cs 30536.4 err 23.1827
 ratio 0.910989 err 0.000769346
 simulated production rate 17.3502
 A 0 exp.cs 217.4 err 1.22
 sim. cs 256.749 err 2.12574
 ratio 1.181 err 0.0118124
 simulated production rate 0.14588
 not found in simulations 0
 not found in exper: 0
 simulated production rate 17.3502
 simulated production rate 0.14588
 matched 2 CHSQ 58.3536
 raw chsq 13620.6
 average ratio 1.04599 err 0.00629088
 lhood 1.14436
 exper. cs 33737.4 err 13.62
 inucl. cs 30793.1 err 25.3085
 ++++++++++++++++++++++++++++++++++++++++++++++++++++++++

 ++++++++++++++++++++++++++++++++++++++++++++++++++++++++
 **** izotop Z **** 1
 A 1 exp.cs 7645 err 6.08
 sim. cs 6801.24 err 10.9408
 ratio 0.889632 err 0.00159645
 simulated production rate 3.86434
 A 0 exp.cs 109.4 err 0.96
 sim. cs 226.195 err 1.99525
 ratio 2.0676 err 0.0257258
 simulated production rate 0.12852
 not found exper.: A 2 exp.cs 658.7 err 4.2
 not found exper.: A 3 exp.cs 338.6 err 3.1
 not found in simulations 2
 not found in exper: 0
 simulated production rate 3.86434
 simulated production rate 0.12852
 matched 2 CHSQ 40.3162
 raw chsq 6501.58
 average ratio 1.47862 err 0.0136611
 lhood 1.68244
 exper. cs 8751.7 err 14.34
 inucl. cs 7027.43 err 12.9361
\end{verbatim}
\normalsize

\section{WORKING DOCUMENT}
\subsection{15.7.2002}

\subsection{11.7.2002}
Now platform is ready. G4 and subcodes are organized and compiles. It is time to do actual interface.
Plan for today work 13-15 and 17-20. Build solid understanding of interface. And build it using tests.
\begin{itemize}
\item to do: ApplyYout self seciton CascadeInterfaceen, koncretic conversion between system nucleust and particles, study more interfacing, write towards working interface
\item KineticTrack is starting point for us.
\item Study G4VIntraNuclearTransport (base class G4HadronicInteraction, G4KineticTrack, G4VParticleChange, G4ReactionProductVector) class. And kinetic model. G4TheoFSGenerator steers the collaboration between hadronic generator and intra-nuclear transport
\item Methods in CascadeInteface :  {\tt G4VParticleChange* ApplyYourself(const G4Track\& aTrack, G4Nucleus\& theNucleus);
   G4ReactionProductVector* Propagate(G4KineticTrackVector* theSecondaries, G4V3DNucleus* theNucleus);}  G4ReactionProductVector ok also for G4VParticleChange
\item For use of ApplyYourself see G4PreCompoundModel
\item
\end{itemize}

\subsection{10.7.2002}
\begin{itemize}
\item {\bf}: localised files (inucl cascade code at {\tt html/inucl/inucl++/demo}, 
hetc code at {\tt html/geant4/geant4/source/processes/ hadronic/models/cascade/cascade},  
documentation {\tt  html/inucl/doc}). 
There is also another g4 version at {\tt html/geant4/geant4.3.1 })
\item studied g4 hadronic cascade interface.
\item {\bf all code and documentation in one place} {\tt html/geant4/geant4/source/processes/ hadronic/models/cascade/cascade}
\item general plan: {\bf thursday} put cascade code to g4 installation (+), make compile (+), study interface structure (+), make small testing (+), 
{\bf friday} create real interface,
{\bf sunday} create inteface, and test basics (modify test13  hadronic testing)
{\bf monday morning at 8} co g4 at cern, pack {\tt hadronic/models/cascade}, replace cascade files, add new ones, and commit. Mail hpw. Start holiday at 10 am.
\item cascading basic structure: {\tt G4Collider (base),
G4ElementaryParticlCollider,
G4IntraNucleiCollider (:G4Collider, has G4ElementaryParticlCollider, ),
G4NucleiModel (initializes cascade,zones, boundary trasitions, fermi energy, density, potential),
G4InucleNuclei/G4HETCNuclei (A, Z, kinetic energy, mass formula, exiton configuration)      }
\item change G4vector to vector:
\item {\tt G4InuclCollider} does the lining of different models inside collision with method command {\tt collide(bull, targ)}
\item {\tt G4IntraNucleiCascader} does the actual cascading. I starts by setting elementary particel cascader {\tt G4IntraNucleiCascader* incascader-> setElementaryParticleCollider( G4ElementaryParticleCollider*)} 
\item The actual cascading is then done: {\tt theIntraNucleiCascader-> collide(\&nbullet, \&ntarget);}
\item finally, laboratory frame change, and some other organizational things happen 
\item g4 inc base class is defined in models/generator/ management/include/ G4VIntraNuclearTransportModel.hh
\item  INC model is set by G4TheoFSGenerator (: public G4HadronicInteraction) by method: {\tt void SetTransport( G4VIntraNuclearTransportModel *const  value);}

\end{itemize}

\subsection{10.6.2002}
\begin{itemize}
\item {\bf compilation}: made compilation faster (now takes 45 seconds) by tunig compiler setting and include-statements
\item {\bf sample run}: result from original C++ test are saved to this document as sample run.
\item {\bf compilation}: fixed writing errors, so code now compiles
\item {\bf manual pages}: tempalte for manual pages copied to this document from GEANT4 ninematic model manual 
\end{itemize}

\subsection{7.6.2002}
\begin{itemize}
\item {\bf document template}: Added new style and collected material to this documet. Rethinking chapter organization.
\item {\bf picture rezise}: {\tt convert x.ps x.eps2}
\item {\bf new references}: Some text added fron new sources
\end{itemize}

\end{appendix}
%\end{document}

% end of file




